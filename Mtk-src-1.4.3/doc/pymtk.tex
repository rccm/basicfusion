\documentclass{howto}
%\usepackage{ltxmarkup}

\title{MISR Toolkit Python Interface}

\makeindex


\begin{document}

\maketitle

\begin{abstract}
\noindent
This document describes the installation and usage of the Python interface to the MISR Toolkit.

\end{abstract}

\section{Installation \label {installation}}
Prerequisites:

  1. \ulink {Python 2.4 or 2.5}{http://www.python.org/}

  2. \ulink {MISR Toolkit}{http://www.openchannelfoundation.org/projects/MISR_Toolkit}

  3. \ulink {NumPy}{http://numpy.scipy.org/} 
 
Building:
\begin{verbatim}
  cd Mtk/wrappers/python
  python setup.py install
\end{verbatim}

\section{Main Classes \label {core}}

\subsection{MtkFile \label{mtkfile}}

\begin{classdesc}{MtkFile}{filename}
  Constructs a new MtkFile object.

\begin{verbatim}
>>> m = MtkFile('../Mtk_testdata/in/MISR_AM1_GRP_ELLIPSOID_GM_P037_O029058_AA_F03_0024.hdf')
\end{verbatim}
\end{classdesc}

\begin{methoddesc}{attr_get}{attribute}
  Get a file attribute.

\begin{verbatim}
>>> m.attr_get('Path_number')
37
\end{verbatim}
\end{methoddesc}

\begin{memberdesc}[list]{attr_list}
  List of file attributes names.

\begin{verbatim}
>>> m.attr_list
['HDFEOSVersion', 'StructMetadata.0', 'Path_number', 'AGP_version_id',
 'DID_version_id', 'Number_blocks', 'Ocean_blocks_size', 'Ocean_blocks.count',
 'Ocean_blocks.numbers', 'SOM_parameters.som_ellipsoid.a',
 'SOM_parameters.som_ellipsoid.e2', 'SOM_parameters.som_orbit.aprime',
 'SOM_parameters.som_orbit.eprime', 'SOM_parameters.som_orbit.gama',
 'SOM_parameters.som_orbit.nrev', 'SOM_parameters.som_orbit.ro',
 'SOM_parameters.som_orbit.i', 'SOM_parameters.som_orbit.P2P1',
 'SOM_parameters.som_orbit.lambda0', 'Origin_block.ulc.x',
 'Origin_block.ulc.y', 'Origin_block.lrc.x', 'Origin_block.lrc.y',
 'Start_block', 'End block', 'Cam_mode', 'Num_local_modes',
 'Local_mode_site_name', 'Orbit_QA', 'Camera', 'coremetadata']
\end{verbatim}
\end{memberdesc}

\begin{memberdesc}[tuple]{block}
  Start and end block numbers.

\begin{verbatim}
>>> m.block
(1, 140)
\end{verbatim}
\end{memberdesc}

\begin{memberdesc}[list]{block_metadata_list}
  List of block metadata structure names.

\begin{verbatim}
>>> m.block_metadata_list
['PerBlockMetadataCommon', 'PerBlockMetadataRad', 'PerBlockMetadataTime']
\end{verbatim}
\end{memberdesc}

\begin{methoddesc}{block_metadata_field_list}{blockmetaname}
  List fields in a block metadata structure.

\begin{verbatim}
>>> m.block_metadata_field_list('PerBlockMetadataCommon')
['Block_number', 'Ocean_flag', 'Block_coor_ulc_som_meter.x',
 'Block_coor_ulc_som_meter.y', 'Block_coor_lrc_som_meter.x',
 'Block_coor_lrc_som_meter.y', 'Data_flag']
\end{verbatim}
\end{methoddesc}
 
\begin{methoddesc}{block_metadata_field_read}{blockmetaname, fieldname}
  Read a block metadata field.

\begin{verbatim}
>>> m.block_metadata_field_read('PerBlockMetadataCommon', 'Block_number')
[1, 2, 3, 4, 5, 6, 7, 8, 9, 10, 11, 12, 13, 14, 15, 16, 17, 18, 19, 20, 21,
 22, 23, 24, 25, 26, 27, 28, 29, 30, 31, 32, 33, 34, 35, 36, 37, 38, 39, 40,
 41, 42, 43, 44, 45, 46, 47, 48, 49, 50, 51, 52, 53, 54, 55, 56, 57, 58, 59,
 60, 61, 62, 63, 64, 65, 66, 67, 68, 69, 70, 71, 72, 73, 74, 75, 76, 77, 78,
 79, 80, 81, 82, 83, 84, 85, 86, 87, 88, 89, 90, 91, 92, 93, 94, 95, 96, 97,
 98, 99, 100, 101, 102, 103, 104, 105, 106, 107, 108, 109, 110, 111, 112, 113,
 114, 115, 116, 117, 118, 119, 120, 121, 122, 123, 124, 125, 126, 127, 128,
 129, 130, 131, 132, 133, 134, 135, 136, 137, 138, 139, 140]
\end{verbatim}
\end{methoddesc}

\begin{methoddesc}{core_metadata_get}{parameter}
  Get core metadata parameter.

\begin{verbatim}
>>> m.core_metadata_get('LOCALGRANULEID')
'MISR_AM1_GRP_ELLIPSOID_GM_P037_O029058_AA_F03_0024.hdf'
\end{verbatim}
\end{methoddesc}

\begin{memberdesc}[list]{core_metadata_list}
  List of core metadata parameter names.

\begin{verbatim}
>>> m.core_metadata_list
['LOCALGRANULEID', 'PRODUCTIONDATETIME', 'LOCALVERSIONID', 'PGEVERSION',
 'MEASUREDPARAMETERCONTAINER', 'AUTOMATICQUALITYFLAGEXPLANATION',
 'AUTOMATICQUALITYFLAG', 'QAPERCENTMISSINGDATA', 'PARAMETERNAME',
 'ORBITCALCULATEDSPATIALDOMAINCONTAINER', 'EQUATORCROSSINGDATE',
 'EQUATORCROSSINGTIME', 'ORBITNUMBER', 'EQUATORCROSSINGLONGITUDE',
 'VERSIONID', 'SHORTNAME', 'INPUTPOINTER', 'GPOLYGONCONTAINER',
 'GRINGPOINTLONGITUDE', 'GRINGPOINTLATITUDE', 'GRINGPOINTSEQUENCENO',
 'EXCLUSIONGRINGFLAG', 'RANGEENDINGDATE', 'RANGEENDINGTIME',
 'RANGEBEGINNINGDATE', 'RANGEBEGINNINGTIME', 'ADDITIONALATTRIBUTESCONTAINER',
 'ADDITIONALATTRIBUTENAME', 'PARAMETERVALUE', 'ADDITIONALATTRIBUTESCONTAINER',
 'ADDITIONALATTRIBUTENAME', 'PARAMETERVALUE', 'ADDITIONALATTRIBUTESCONTAINER',
 'ADDITIONALATTRIBUTENAME', 'PARAMETERVALUE', 'ADDITIONALATTRIBUTESCONTAINER',
 'ADDITIONALATTRIBUTENAME', 'PARAMETERVALUE',
 'ASSOCIATEDPLATFORMINSTRUMENTSENSORCONTAINER',
 'ASSOCIATEDSENSORSHORTNAME', 'ASSOCIATEDPLATFORMSHORTNAME', 'OPERATIONMODE',
 'ASSOCIATEDINSTRUMENTSHORTNAME']
\end{verbatim}
\end{memberdesc}

\begin{memberdesc}[char]{file_name}
  File name of file.

\begin{verbatim}
>>> m.file_name
'../Mtk_testdata/in/MISR_AM1_GRP_ELLIPSOID_GM_P037_O029058_AA_F03_0024.hdf'
\end{verbatim}
\end{memberdesc}

\begin{memberdesc}[char]{file_type}
  MISR product file type.

\begin{verbatim}
>>> m.file_type
'GRP_ELLIPSOID_GM'
\end{verbatim}
\end{memberdesc}

\begin{methoddesc}{grid}{grid_name}
  Returns MtkGrid object for grid_name.

\begin{verbatim}
>>> m.grid('BlueBand')
<MisrToolkit.MtkGrid object at 0x140e740>
\end{verbatim}
\end{methoddesc}

\begin{memberdesc}[list]{grid_list}
  List of grid names.

\begin{verbatim}
>>> m.grid_list
['BlueBand', 'GreenBand', 'RedBand', 'NIRBand', 'BRF Conversion Factors', 'GeometricParameters']
\end{verbatim}
\end{memberdesc}

\begin{memberdesc}[char]{local_granule_id}
  Local granual ID of MISR product file.

\begin{verbatim}
>>> m.local_granule_id
'MISR_AM1_GRP_ELLIPSOID_GM_P037_O029058_AA_F03_0024.hdf'
\end{verbatim}
\end{memberdesc}

\begin{memberdesc}[int]{orbit}
  Orbit number.

\begin{verbatim}
>>> m.orbit
29058
\end{verbatim}
\end{memberdesc}

\begin{memberdesc}[int]{path}
  Path number.

\begin{verbatim}
>>> m.path
37
\end{verbatim}
\end{memberdesc}

\begin{methoddesc}{time_metadata_read}{}
  Read time metadata from L1B2 Ellipsoid product.

\begin{verbatim}
>>> m.time_metadata_read()
<MisrToolkit.MtkTimeMetaData object at 0x158f000>
\end{verbatim}

\begin{notice}[note]
Time metadata is stored in the L1B2 Ellipsoid product, it is not available in any other product.
\end{notice}

\end{methoddesc}

\begin{memberdesc}[char]{version}
  MISR product file version.

\begin{verbatim}
>>> m.version
'F03_0024'
\end{verbatim}
\end{memberdesc}


\subsection{MtkGrid \label{mtkgrid}}

\begin{classdesc*}{MtkGrid}
  Grid from file.

\begin{verbatim}
>>> g = MtkFile('../Mtk_testdata/in/MISR_AM1_GRP_ELLIPSOID_GM_P037_O029058_AA_F03_0024.hdf').grid('BlueBand')
\end{verbatim}
\end{classdesc*}

\begin{methoddesc}{attr_get}{attr_name}
  Get a grid attribute.

\begin{verbatim}
>>> g.attr_get('Block_size.resolution_x')
1100
\end{verbatim}
\end{methoddesc}

\begin{memberdesc}[char]{attr_list}
  List of attribute names.

\begin{verbatim}
>>> g.attr_list
['Block_size.resolution_x', 'Block_size.resolution_y', 'Block_size.size_x', 'Block_size.size_y', 'Scale factor', 'std_solar_wgted_height', 'SunDistanceAU']
\end{verbatim}
\end{memberdesc}

\begin{methoddesc}{field_dims}{field_name}
  Returns a list of tuples of the extra dimension names and sizes.  If field_name 
  doesn't have extra dimensions an empty list is returned.

\begin{verbatim}
>>> g.field_dims('Blue Radiance')
[]
\end{verbatim}
\end{methoddesc}

\begin{methoddesc}{field}{field_name}
  Return MtkField.

\begin{verbatim}
>>> g.field('Blue Radiance')
<MisrToolkit.MtkField object at 0x15137a0>
\end{verbatim}
\end{methoddesc}

\begin{memberdesc}[char]{field_list}
  List of field names.

\begin{verbatim}
>>> g.field_list
['Blue Radiance/RDQI', 'Blue Radiance', 'Blue RDQI', 'Blue DN', 'Blue Equivalent Reflectance', 'Blue Brf']
\end{verbatim}
\end{memberdesc}

\begin{memberdesc}[char]{native_field_list}
  List of native field names (excludes derived fields).

\begin{verbatim}
>>> g.native_field_list
['Blue Radiance/RDQI']
\end{verbatim}
\end{memberdesc}

\begin{memberdesc}[char]{grid_name}
  Grid name.

\begin{verbatim}
>>> g.grid_name
'BlueBand'
\end{verbatim}
\end{memberdesc}

\begin{memberdesc}[int]{resolution}
  Resolution of grid in meters.

\begin{verbatim}
>>> g.resolution
1100
\end{verbatim}
\end{memberdesc}


\subsection{MtkField \label{mtkfield}}

\begin{classdesc*}{MtkField}{fieldname}
  Field from grid.

\begin{verbatim}
>>> f = MtkFile('../Mtk_testdata/in/MISR_AM1_GRP_ELLIPSOID_GM_P037_O029058_AA_F03_0024.hdf').grid('BlueBand').field('Blue Radiance')
\end{verbatim}
\end{classdesc*}

\begin{memberdesc}[char]{data_type}
  Data type of field.

\begin{verbatim}
>>> f.data_type
'uint16'
\end{verbatim}
\end{memberdesc}

\begin{memberdesc}[char]{field_name}
  Field name.

\begin{verbatim}
>>> f.field_name
'Blue Radiance'
\end{verbatim}
\end{memberdesc}

\begin{memberdesc}[]{fill_value}
  Fill value.

\begin{verbatim}
>>> f.fill_value
65515
\end{verbatim}
\end{memberdesc}

\begin{methoddesc}{read}{region}
  Read data from field by specifying a MtkRegion.  Returns a MtkDataPlane.

\begin{verbatim}
>>> r = MtkRegion(37, 50, 55)
>>> f.read(r)
<MisrToolkit.MtkDataPlane object at 0x18a0400>
\end{verbatim}

\begin{notice}[note]
   The MtkField read method always return a 2-D data plane buffer.  Some fields in the MISR data products
   are multi-dimensional.  In order to read one of these fields, the slice to read needs to be specified.
   A bracket notation on the fieldname is used for this purpose.  For example RetrAppMask[0][5].

   Additional dimensions can be determined using MtkGrid field_dims method or by referencing MISR Data
   Product Specification (DPS) Document.  The actually definition of the indices are not described in the
   MISR product files and thus not described by the MISR Toolkit.  These will have to be looked up in the
   MISR DPS.  All indices are 0-based.
\end{notice}

\end{methoddesc}

\begin{methoddesc}{read}{start_block, end_block}
  Reads native fields and returns a 3-D NumPy array for the block range. The
  blocks are not assembled and are just stacked on top of each other. The
  CoordQuery functions can be used to map or convert geographic coordinates
  into block, line and sample which correspond to the data returned by this
  function.

  This function is provided as an alternative to reading blocks one at a time
  using a MtkRegion.

\begin{verbatim}
>>> f.read(50, 55)
array([[[65515, 65515, 65515, ..., 65515, 65515, 65515],
        [65515, 65515, 65515, ..., 65515, 65515, 65515],
        [65515, 65515, 65515, ..., 65515, 65515, 65515],
        ...,
        [65515, 65515, 65515, ..., 65515, 65515, 65515],
        [65515, 65515, 65515, ..., 65515, 65515, 65515],
        [65515, 65515, 65515, ..., 65515, 65515, 65515]]], dtype=uint16)
\end{verbatim}

\begin{notice}[note]
  The block index returned by the CoordQuery functions are 1-based
  and are referenced to the entire MISR path. The 3-D array returned by this
  method is referenced to your 1-based start block and Python uses 0-based
  indexing, so adjust the block index accordingly.
\end{notice}
\end{methoddesc}


\subsection{MtkRegion \label{mtkregion}}

\begin{classdesc}{MtkRegion}{}
  Construct a new MtkRegion object.

\begin{verbatim}
>>> r = MtkRegion()
\end{verbatim}
\end{classdesc}

\begin{classdesc}{MtkRegion}{path, start_block, end_block}
  Construct a new MtkRegion object by path and block range.

\begin{verbatim}
>>>  r = MtkRegion(37, 50, 60)
\end{verbatim}
\end{classdesc}

\begin{classdesc}{MtkRegion}{ulc_lat, ulc_lon, lrc_lat, lrc_lon}
  Construct a new MtkRegion object by upper left corner, lower right corner.

\begin{verbatim}
>>> r = MtkRegion(40.0, -120.0, 30.0, -110.0)
\end{verbatim}
\end{classdesc}

\begin{classdesc}{MtkRegion}{ctr_lat, ctr_lon, lat_extent, lon_extent, extent_units}
  Construct a new MtkRegion object by latitude, longitude in decimal degrees, and extent in specified units.

  The extent_units argument is a case insensitive string that can be set to one of the following values:

   1. "degrees", "deg", "dd" for degrees;

   2. "meters", "m" for meters;

   3. "kilometers", "km" for kilometers; and

   4. "275m", "275 meters", "1.1km", "1.1 kilometers" for pixels of a specified resolution per pixel.

\begin{verbatim}
>>> r = MtkRegion(35.0, -115.0, 1.5, 2.0, "deg")
>>> r = MtkRegion(35.0, -115.0, 5000.0, 8000.0, "m")
>>> r = MtkRegion(35.0, -115.0, 2.2, 1.1, "km")
>>> r = MtkRegion(35.0, -115.0, 45.0, 100.0, "275m")
>>> r = MtkRegion(35.0, -115.0, 35.0, 25.0, "1.1km")
\end{verbatim}
\end{classdesc}

\begin{classdesc}{MtkRegion}{path, ulc_som_x, ulc_som_y, lrc_som_x, lrc_som_y}
  Construct a new MtkRegion object by Path and SOM X/Y of upper left corner and lower right corner in meters. 

\begin{verbatim}
>>> r = MtkRegion(27, 15600000.0, -300.0, 16800000.0, 2000.0)
\end{verbatim}
\end{classdesc}

\begin{methoddesc}{block_range}{path}
  Return block range that covers the region for the given path.

\begin{verbatim}
>>> r.block_range(37)
(59, 67)
\end{verbatim}
\end{methoddesc}

\begin{memberdesc}[tuple]{center}
  Center coordinate of the region in degrees.

\begin{verbatim}
>>> r.center
(35.0, -115.0)
\end{verbatim}
\end{memberdesc}

\begin{memberdesc}[tuple]{extent}
  Extent of the region in meters.

\begin{verbatim}
>>> r.extent
(1113195.4314, 1113195.4314)
\end{verbatim}
\end{memberdesc}

\begin{memberdesc}[list]{path_list}
  List of paths that cover the region.

\begin{verbatim}
>>> r.path_list
[33, 34, 35, 36, 37, 38, 39, 40, 41, 42, 43, 44, 45]
\end{verbatim}
\end{memberdesc}

\begin{methoddesc}{snap_to_grid}{path, resolution}
  Snap a region to a MISR grid based on path number and resolution.

\begin{verbatim}
>>> r.snap_to_grid(37, 1100)
<MisrToolkit.MtkMapInfo object at 0x1897a00>
\end{verbatim}
\end{methoddesc}


\subsection{MtkDataPlane \label{mtkdataplane}}

\begin{classdesc*}{MtkDataPlane}
  Contains data and map information.

\begin{verbatim}
>>> r = MtkRegion(37, 50, 50)
>>> d = MtkFile('../Mtk_testdata/in/MISR_AM1_GRP_ELLIPSOID_GM_P037_O029058_AA_F03_0024.hdf').grid('BlueBand').field('Blue Radiance').read(r)
\end{verbatim}
\end{classdesc*}

\begin{methoddesc}{data}{}
  Returns a NumPy of the data in the plane.

\begin{verbatim}
>>> d.data()
array([[ 0.,  0.,  0., ...,  0.,  0.,  0.],
       [ 0.,  0.,  0., ...,  0.,  0.,  0.],
       [ 0.,  0.,  0., ...,  0.,  0.,  0.],
       ..., 
       [ 0.,  0.,  0., ...,  0.,  0.,  0.],
       [ 0.,  0.,  0., ...,  0.,  0.,  0.],
       [ 0.,  0.,  0., ...,  0.,  0.,  0.]], dtype=float32)
\end{verbatim}
\end{methoddesc}

\begin{methoddesc}{mapinfo}{}
  Returns a MtkMapInfo for the data in the plane.

\begin{verbatim}
>>> d.mapinfo()
<MisrToolkit.MtkMapInfo object at 0x189b200>
\end{verbatim}
\end{methoddesc}


\subsection{MtkMapInfo \label{mtkmapinfo}}

\begin{classdesc*}{MtkMapInfo}
  Contains map information, and supports map queries.
  
\begin{verbatim}
>>> map_info = MtkRegion(37, 35, 36).snap_to_grid(37, 1100)
\end{verbatim}
\end{classdesc*}

\begin{methoddesc}{create_latlon}{}
    Create a latitude array and a longitude array in decimal degrees.

\begin{verbatim}
>>> map_info.create_latlon()
(array([[ 69.84396725,  69.84125375,  69.83853597, ...,  67.95214383,
         67.94751056,  67.94287408],
       [ 69.83449667,  69.83178431,  69.82906768, ...,  67.94344755,
         67.93881609,  67.9341814 ],
       [ 69.82502575,  69.82231453,  69.81959905, ...,  67.93475031,
         67.93012064,  67.92548775],
       ..., 
       [ 67.43810121,  67.43564618,  67.43318732, ...,  65.72400133,
         65.71978632,  65.71556829],
       [ 67.42855667,  67.42610255,  67.4236446 , ...,  65.71509301,
         65.71087952,  65.70666301],
       [ 67.41901189,  67.41655868,  67.41410164, ...,  65.70618396,
         65.701972  ,  65.697757  ]]),
         
 array([[ -99.94798275,  -99.92052402,  -99.89307216, ...,  -87.01432073,
         -86.9911815 ,  -86.96805169],
       [ -99.95583948,  -99.92839213,  -99.90095165, ...,  -87.0266458 ,
         -87.00351269,  -86.98038898],
       [ -99.96368936,  -99.93625339,  -99.90882428, ...,  -87.03896147,
         -87.01583448,  -86.99271687],
       ..., 
       [-101.7397708 , -101.71492204, -101.69007822, ...,  -89.85775933,
         -89.83612005,  -89.81448799],
       [-101.74615355, -101.72131411, -101.69647961, ...,  -89.86799566,
         -89.84636208,  -89.82473572],
       [-101.75253136, -101.72770125, -101.70287606, ...,  -89.87822477,
         -89.85659689,  -89.83497623]]))
\end{verbatim}
\end{methoddesc}

\begin{memberdesc}[int]{end_block}
  End block number.

\begin{verbatim}
>>> map_info.end_block
36
\end{verbatim}
\end{memberdesc}

\begin{memberdesc}[MtkGeoRegion]{geo}
  MtkGeoRegion object.

\begin{verbatim}
>>> map_info.geo
<MisrToolkit.MtkGeoRegion object at 0xe458>
>>> map_info.geo.ctr
(67.848508,-94.442838)
\end{verbatim}
\end{memberdesc}

\begin{methoddesc}{latlon_to_ls}{lat, lon}
    Lat and Lon to Line Sample.

\begin{verbatim}
>>> map_info.latlon_to_ls(68.36,-97.74)             
(120.52, 120.55)
\end{verbatim}
\end{methoddesc}

\begin{methoddesc}{ls_to_latlon}{line, sample}
    Line and Sample To Lat and Lon.

\begin{verbatim}
>>> map_info.ls_to_latlon(120,120)
(68.36, -97.74)
\end{verbatim}
\end{methoddesc}

\begin{methoddesc}{ls_to_somxy}{line, sample}
    Convert Line and Sample values to Som X and Som Y coordinates.  line and sample can be either scalar values or numarrays.

\begin{verbatim}
>>> map_info.ls_to_somxy(120,120) 
(12380500.0, 677600.0)
\end{verbatim}
\end{methoddesc}

\begin{memberdesc}[int]{nline}
  Number of lines.

\begin{verbatim}
>>> map_info.nline                            
256
\end{verbatim}
\end{memberdesc}

\begin{memberdesc}[int]{nsample}
  Number of samples.

\begin{verbatim}
>>> map_info.nsample
512
\end{verbatim}
\end{memberdesc}

\begin{memberdesc}[int]{path}
  Path number.
  
\begin{verbatim}
>>> map_info.path   
37
\end{verbatim}
\end{memberdesc}

\begin{memberdesc}[PyBool]{pixelcenter}
  Pixel Center.

\begin{verbatim}
>>> map_info.pixelcenter
True
\end{verbatim}
\end{memberdesc}

\begin{memberdesc}[MtkProjParam]{pp}
  MtkProjParam object.

\begin{verbatim}
>>> map_info.pp         
<MisrToolkit.MtkProjParam object at 0x18ade00>
\end{verbatim}
\end{memberdesc}

\begin{memberdesc}[int]{resfactor}
  Resfactor.

\begin{verbatim}
>>> map_info.resfactor
4
\end{verbatim}
\end{memberdesc}

\begin{memberdesc}[int]{resolution}
  Resolution.

\begin{verbatim}
>>> map_info.resolution
1100
\end{verbatim}
\end{memberdesc}

\begin{memberdesc}[MtkSomRegion]{som}
  MtkSomRegion object.

\begin{verbatim}
>>> map_info.som       
<MisrToolkit.MtkSomRegion object at 0xe338>
>>> map_info.som.ctr
(12388750.0,826650.0)
\end{verbatim}
\end{memberdesc}

\begin{methoddesc}{somxy_to_ls}{somx, somy}
    Som X and Som Y To Line and Sample.

\begin{verbatim}
>>> map_info.somxy_to_ls(12380500.0, 677600.0)
(120.0, 120.0)
\end{verbatim}
\end{methoddesc}

\begin{memberdesc}[int]{start_block}
  Start block number.

\begin{verbatim}
>>> map_info.start_block
35
\end{verbatim}
\end{memberdesc}


\subsection{MtkTimeMetaData \label{mtktimemetadata}}


\begin{classdesc*}{MtkTimeMetaData}
  Contains time metadata information, and supports calculating pixel time.
  
\begin{verbatim}
>>> time_metadata = MtkFile('../Mtk_testdata/in/MISR_AM1_GRP_ELLIPSOID_GM_P037_O029058_AA_F03_0024.hdf').time_metadata_read()
\end{verbatim}

\begin{notice}[note]
Time metadata is stored in the L1B2 Ellipsoid product, it is not available in any other product. The pixel time varies according to camera by approximately 7 minutes. To get the average or center pixel acquistion time, it is recommended to use the time metadata from the AN camera.
\end{notice}

\end{classdesc*}

\begin{memberdesc}[char]{camera}
  Camera for which time metadata applies.
  
\begin{verbatim}
>>> time_metadata.camera  
'AA'
\end{verbatim}
\end{memberdesc}

\begin{memberdesc}[double]{coeff_line}
  Line transform coefficients.
  
\begin{verbatim}
>>> time_metadata.coeff_line
array([[[  0.00000000e+00,   0.00000000e+00],
        [  0.00000000e+00,   0.00000000e+00],
        [  0.00000000e+00,   0.00000000e+00],
        [  0.00000000e+00,   0.00000000e+00],
        [  0.00000000e+00,   0.00000000e+00],
        [  0.00000000e+00,   0.00000000e+00]],

       [[  2.06016807e+03,   2.31347217e+03],
        [  9.89420163e-01,   9.89470448e-01],
        [  2.32523204e-02,   2.23010430e-02],
        [  1.03023724e-05,   1.03026080e-05],
        [ -3.72946711e-06,  -3.72871997e-06],
        [ -6.93100473e-13,  -1.58848849e-11]],

       [[  2.56679833e+03,   2.82013585e+03],
        [  9.89537006e-01,   9.89609347e-01],
        [  2.13693245e-02,   2.04218034e-02],
        [  1.03134648e-05,   1.03101703e-05],
        [ -3.77586865e-06,  -3.69789446e-06],
        [ -5.35620372e-11,  -1.11540278e-11]],

       ..., 
       [[  0.00000000e+00,   0.00000000e+00],
        [  0.00000000e+00,   0.00000000e+00],
        [  0.00000000e+00,   0.00000000e+00],
        [  0.00000000e+00,   0.00000000e+00],
        [  0.00000000e+00,   0.00000000e+00],
        [  0.00000000e+00,   0.00000000e+00]],

       [[  0.00000000e+00,   0.00000000e+00],
        [  0.00000000e+00,   0.00000000e+00],
        [  0.00000000e+00,   0.00000000e+00],
        [  0.00000000e+00,   0.00000000e+00],
        [  0.00000000e+00,   0.00000000e+00],
        [  0.00000000e+00,   0.00000000e+00]],

       [[  0.00000000e+00,   0.00000000e+00],
        [  0.00000000e+00,   0.00000000e+00],
        [  0.00000000e+00,   0.00000000e+00],
        [  0.00000000e+00,   0.00000000e+00],
        [  0.00000000e+00,   0.00000000e+00],
        [  0.00000000e+00,   0.00000000e+00]]])
\end{verbatim}
\end{memberdesc}

\begin{memberdesc}[int]{end_block}
  End block number.

\begin{verbatim}
>>> time_metadata.end_block 
140
\end{verbatim}
\end{memberdesc}

\begin{memberdesc}[int]{number_line}
  Number of lines.
  
\begin{verbatim}
>>> time_metadata.number_line
array([[  0,   0],
       [256, 256],
       [256, 256],
       [256, 256],
       ...,
       [  0,   0],
       [  0,   0],
       [  0,   0],
       [  0,   0]])
       
\end{verbatim}
\end{memberdesc}

\begin{memberdesc}[int]{number_transform}
  Number of transforms.
  
\begin{verbatim}
>>> time_metadata.number_transform
array([0, 2, 2, 2, 2, 2, 2, 2, 2, 2, 2, 2, 2, 2, 2, 2, 2, 2, 2, 2, 2, 2, 2,
       2, 2, 2, 2, 2, 2, 2, 2, 2, 2, 2, 2, 2, 2, 2, 2, 2, 2, 2, 2, 2, 2, 2,
       2, 2, 2, 2, 2, 2, 2, 2, 2, 2, 2, 2, 2, 2, 2, 2, 2, 2, 2, 2, 2, 2, 2,
       2, 2, 2, 2, 2, 2, 2, 2, 2, 2, 2, 2, 2, 2, 2, 2, 2, 2, 2, 2, 2, 2, 2,
       2, 2, 2, 2, 2, 2, 2, 2, 2, 2, 2, 2, 2, 2, 2, 2, 2, 2, 2, 2, 2, 2, 2,
       2, 2, 2, 2, 2, 2, 2, 2, 2, 2, 2, 2, 2, 2, 2, 2, 2, 2, 2, 2, 2, 2, 2,
       2, 2, 1, 0, 0, 0, 0, 0, 0, 0, 0, 0, 0, 0, 0, 0, 0, 0, 0, 0, 0, 0, 0,
       0, 0, 0, 0, 0, 0, 0, 0, 0, 0, 0, 0, 0, 0, 0, 0, 0, 0, 0, 0])
\end{verbatim}
\end{memberdesc}

\begin{memberdesc}[int]{path}
  Path number.
  
\begin{verbatim}
>>> time_metadata.path
37
\end{verbatim}
\end{memberdesc}

\begin{methoddesc}{pixel_time}{som_x, som_y}
  Calculate pixel time at Som X, Som Y.

\begin{verbatim}
>>> time_metadata.pixel_time(10153687.5, 738787.5)
'2005-06-04T18:06:07.656501Z'
\end{verbatim}
\end{methoddesc}

\begin{memberdesc}[char]{ref_time} 
  Reference time.

\begin{verbatim}
>>> time_metadata.ref_time
[['', ''], ['2005-06-04T17:58:13.127920Z', '2005-06-04T17:58:13.127920Z'], ... ['', ''], ['', '']]
\end{verbatim}
\end{memberdesc}

\begin{memberdesc}[double]{som_ctr_x}
  SOM X center coordinates.

\begin{verbatim}
>>> time_metadata.som_ctr_x
array([[     0.,      0.],
       [   128.,    384.],
       [   640.,    896.],
       ...
       [     0.,      0.],
       [     0.,      0.],
       [     0.,      0.]])
\end{verbatim}
\end{memberdesc}

\begin{memberdesc}[double]{som_ctr_y}
  SOM Y center coordinates.

\begin{verbatim}
>>> time_metadata.som_ctr_y
array([[    0.,     0.],
       [ 1024.,  1024.],
       [ 1024.,  1024.],
       ...
       [     0.,      0.],
       [     0.,      0.],
       [     0.,      0.]])
\end{verbatim}
\end{memberdesc}

\begin{memberdesc}[int]{start_block}
  Start block number.

\begin{verbatim}
>>> time_metadata.start_block
1
\end{verbatim}
\end{memberdesc}

\begin{memberdesc}[int]{start_line}
  Starting line.

\begin{verbatim}
>>> time_metadata.start_line 
array([[    0,     0],
       [    0,   256],
       [  512,   768],
       ...
       [    0,     0],
       [    0,     0],
       [    0,     0]])
\end{verbatim}
\end{memberdesc}


\subsection{MtkReProject \label{mtkreproject}}

\begin{classdesc*}{MtkReProject}
  Contains ReProjection Functionality.

\begin{verbatim}
>>> reproj = MtkReProject()
\end{verbatim}
\end{classdesc*}

\begin{methoddesc}{create_geogrid}{}
    Create a geo-grid array in decimal degress for lat/lon.

\begin{verbatim}
>>> reproj.create_geogrid(40,-120,30,-110,0.25,0.25)
(array([[ 40.  ,  40.  ,  40.  , ...,  40.  ,  40.  ,  40.  ],
       [ 39.75,  39.75,  39.75, ...,  39.75,  39.75,  39.75],
       [ 39.5 ,  39.5 ,  39.5 , ...,  39.5 ,  39.5 ,  39.5 ],
       ...,
       [ 30.5 ,  30.5 ,  30.5 , ...,  30.5 ,  30.5 ,  30.5 ],
       [ 30.25,  30.25,  30.25, ...,  30.25,  30.25,  30.25],
       [ 30.  ,  30.  ,  30.  , ...,  30.  ,  30.  ,  30.  ]]), 
       
array([[-120.  , -119.75, -119.5 , ..., -110.5 , -110.25, -110.  ],
       [-120.  , -119.75, -119.5 , ..., -110.5 , -110.25, -110.  ],
       [-120.  , -119.75, -119.5 , ..., -110.5 , -110.25, -110.  ],
       ...,
       [-120.  , -119.75, -119.5 , ..., -110.5 , -110.25, -110.  ],
       [-120.  , -119.75, -119.5 , ..., -110.5 , -110.25, -110.  ],
       [-120.  , -119.75, -119.5 , ..., -110.5 , -110.25, -110.  ]]))
\end{verbatim}
\end{methoddesc}

\begin{methoddesc}{resample_cubic_convolution}{}
  Resample source data at the given coordinates using interpolation by cubic convolution.

\begin{verbatim}
>>> srcdata = numpy.ones((128,512), dtype=numpy.float32) * 0.04
>>> datashape = srcdata.shape
>>> srcmask = numpy.ones(datashape, dtype=numpy.uint8)
>>> a = -0.5;
>>> regrshape = tuple((float(dimen) * abs(a)) for dimen in datashape)
>>> lines = numpy.tile(numpy.linspace(4.1,((10*regrshape[0]) + 4.1), regrshape[1]), (regrshape[0],1)).astype('float32')
>>> samples = numpy.tile(numpy.linspace(4.1,((10*regrshape[1]) + 4.1), regrshape[0]), (regrshape[1],1)).transpose().astype('float32')
>>> myproj.resample_cubic_convolution(srcdata, srcmask, lines, samples, a )
(array([[ 0.  ,  0.  ,  0.  , ...,  0.  ,  0.  ,  0.  ],
       [ 0.04,  0.04,  0.04, ...,  0.04,  0.04,  0.04],
       [ 0.04,  0.04,  0.04, ...,  0.04,  0.04,  0.04],
       ...,
       [ 0.04,  0.04,  0.04, ...,  0.04,  0.04,  0.04],
       [ 0.04,  0.04,  0.04, ...,  0.04,  0.04,  0.04],
       [ 0.04,  0.04,  0.04, ...,  0.04,  0.04,  0.04]], dtype=float32),
       
array([[0, 0, 0, ..., 0, 0, 0],
       [1, 1, 1, ..., 1, 1, 1],
       [1, 1, 1, ..., 1, 1, 1],
       ...,
       [1, 1, 1, ..., 1, 1, 1],
       [1, 1, 1, ..., 1, 1, 1],
       [1, 1, 1, ..., 1, 1, 1]], dtype=int8))
\end{verbatim}
\end{methoddesc}

\begin{methoddesc}{resample_nearest_neighbor}{}
    Performs nearest neighbor resampling.

\begin{verbatim}
>>> srcdata = numpy.ones((128,512), dtype=numpy.float32) * 20
>>> datashape = srcdata.shape
>>> regrshape = tuple( (float(dimen) * 0.5) for dimen in datashape)
>>> lines = numpy.tile(numpy.linspace(4.1,((10*regrshape[0]) + 4.1), regrshape[1]), (regrshape[0],1)).astype('float32')
>>> samples = numpy.tile(numpy.linspace(4.1,((10*regrshape[1]) + 4.1), regrshape[0]), (regrshape[1],1)).transpose().astype('float32')
>>> myproj.resample_nearest_neighbor(srcdata, lines, samples)
array([[ 0.,  0.,  0., ...,  0.,  0.,  0.],
       [ 0.,  0.,  0., ...,  0.,  0.,  0.],
       [ 0.,  0.,  0., ...,  0.,  0.,  0.],
       ...,
       [ 0.,  0.,  0., ...,  0.,  0.,  0.],
       [ 0.,  0.,  0., ...,  0.,  0.,  0.],
       [ 0.,  0.,  0., ...,  0.,  0.,  0.]], dtype=float32)
\end{verbatim}
\end{methoddesc}

\begin{methoddesc}{transform_coordinates}{}
  Transforms latitude/longitude coordinates into line/sample coordinates.

\begin{verbatim}
>>> map_info = MtkRegion(39, 51, 52).snap_to_grid(39,275)
>>> latbuf, lonbuf = map_info.create_latlon()
>>> myproj.transform_coordinates(map_info, latbuf, lonbuf)
(array([[-1., -1., -1., ..., -1., -1., -1.],
       [-1., -1., -1., ..., -1., -1., -1.],
       [-1., -1., -1., ..., -1., -1., -1.],
       ...,
       [-1., -1., -1., ..., -1., -1., -1.],
       [-1., -1., -1., ..., -1., -1., -1.],
       [-1., -1., -1., ..., -1., -1., -1.]], dtype=float32), array([[-1., -1., -1., ..., -1., -1., -1.],
       [-1., -1., -1., ..., -1., -1., -1.],
       [-1., -1., -1., ..., -1., -1., -1.],
       ...,
       [-1., -1., -1., ..., -1., -1., -1.],
       [-1., -1., -1., ..., -1., -1., -1.],
       [-1., -1., -1., ..., -1., -1., -1.]], dtype=float32))
\end{verbatim}
\end{methoddesc}

\subsection{MtkRegression \label{mtkregression}}

\begin{classdesc*}{MtkRegression}
  Contains Regression Functionality.

\begin{verbatim}
>>> regr = MtkRegression()
\end{verbatim}
\end{classdesc*}

\begin{methoddesc}{downsample}{}
    Downsamples data by averaging pixels.

\begin{verbatim}
>>> r = MtkRegion(37, 50, 60)
>>> m = MtkFile('../Mtk_testdata/in/MISR_AM1_GRP_ELLIPSOID_GM_P037_O029058_AA_F03_0024.hdf')
>>> g = m.grid('BlueBand')
>>> f = m.grid('BlueBand').field('Blue Radiance')
>>> srcdata = f.read(r).data()
>>> srcmask = numpy.ones((srcdata.shape[0],srcdata.shape[1]), dtype= numpy.uint8)
>>> sizefactor = 2
>>> rsmpdata, rsmpmask = regr.downsample(srcdata,srcmask,sizefactor)
>>> print rsmpdata[0][90]
315.341
\end{verbatim}
\end{methoddesc}

\begin{methoddesc}{linear_regression_calc}{}
    Uses linear regression to fit data.

\begin{verbatim}
>>> x = numpy.linspace(2,12,num = 5)
>>> y = numpy.linspace(4,24,num = 5)
>>> ysigma = numpy.linspace(0.1,0.6,num = 5)
>>> regr.linear_regression_calc(x,y,ysigma)
(array([  0.00000000e+000,  -2.32035723e+077,   2.15126649e-314,
         1.06717989e-287,   3.38503263e+125]), 
array([  2.00000000e+000,  -2.68679011e+154,   2.15168741e-314,
         2.15170772e-314,   2.32035723e+077]), 
array([  1.00000000e+000,   3.11109040e+231,   2.15175217e-314,
         6.94154000e-310,  -3.11109040e+231]))
\end{verbatim}
\end{methoddesc}

\begin{methoddesc}{smooth_data}{}
    Smooths the given array with a boxcar average.

\begin{verbatim}
>>> r = MtkRegion(37, 50, 60)
>>> m = MtkFile('../Mtk_testdata/in/MISR_AM1_GRP_ELLIPSOID_GM_P037_O029058_AA_F03_0024.hdf')
>>> g = m.grid('BlueBand')
>>> f = m.grid('BlueBand').field('Blue Radiance')
>>> srcdata = f.read(r).data()
>>> srcmask = numpy.ones((srcdata.shape[0],srcdata.shape[1]), dtype= numpy.uint8)
>>> line_width = 3
>>> sample_width = 3
>>> regr.smooth_data(srcdata,srcmask,line_width,sample_width)
array([[ 0.,  0.,  0., ...,  0.,  0.,  0.],
       [ 0.,  0.,  0., ...,  0.,  0.,  0.],
       [ 0.,  0.,  0., ...,  0.,  0.,  0.],
       ...,
       [ 0.,  0.,  0., ...,  0.,  0.,  0.],
       [ 0.,  0.,  0., ...,  0.,  0.,  0.],
       [ 0.,  0.,  0., ...,  0.,  0.,  0.]], dtype=float32)
\end{verbatim}
\end{methoddesc} 

\begin{methoddesc}{upsample_mask}{}
    Upsamples a mask by nearest neighbor sampling.

\begin{verbatim}
>>> srcmask = numpy.ones((1408,624), dtype= numpy.uint8)
>>> size_factor = 2
>>> regr.upsample_mask(srcmask,size_factor)
array([[1, 1, 1, ..., 1, 1, 1],
       [1, 1, 1, ..., 1, 1, 1],
       [0, 0, 0, ..., 0, 0, 0],
       ...,
       [0, 0, 0, ..., 0, 0, 0],
       [0, 0, 0, ..., 0, 0, 0],
       [0, 0, 0, ..., 0, 0, 0]], dtype=uint8) 
\end{verbatim}
\end{methoddesc}

\begin{methoddesc}{coeff_calc}{}
    Calculates linear regression coefficients for translating values.

\begin{verbatim}
>>> r = MtkRegion(37, 50, 60)
>>> m = MtkFile('../Mtk_testdata/in/MISR_AM1_GRP_ELLIPSOID_GM_P037_O029058_AA_F03_0024.hdf')
>>> g1 = m.grid('BlueBand')
>>> f1 = g1.field('Blue Radiance')
>>> data1 = f1.read(r).data()
>>> mask1 = numpy.ones((data1.shape[0],data1.shape[1]), dtype= numpy.uint8)
>>> g2 = m.grid('GreenBand')
>>> f2 = g2.field('Green Radiance')
>>> data2 = f2.read(r).data()
>>> sigma2 = numpy.tile((numpy.linspace(0.1,0.6,data2.shape[0])), (data2.shape[1],1)).transpose()
>>> sigma2 = sigma2.astype(numpy.float32)
>>> mask2 = numpy.ones((data2.shape[0],data2.shape[1]), dtype= numpy.uint8)
>>> mapinfo = f2.read(r).mapinfo()
>>> sizefactor = 2
>>> regr.coeff_calc(data1, mask1, data2, sigma2, mask2, sizefactor, mapinfo)
(<MisrToolkit.MtkRegCoeff object at 0x102e99930>, <MisrToolkit.MtkMapInfo object at 0x7fde19304400>)
\end{verbatim}
\end{methoddesc} 

\begin{methoddesc}{apply_regression}{}
    Applies regression to given data.

\begin{verbatim}
>>> r = MtkRegion(37, 50, 60)
>>> m = MtkFile('../Mtk_testdata/in/MISR_AM1_GRP_ELLIPSOID_GM_P037_O029058_AA_F03_0024.hdf')
>>> g1 = m.grid('BlueBand')
>>> f1 = g1.field('Blue Radiance')
>>> data1 = f1.read(r).data()
>>> mask1 = numpy.ones((data1.shape[0],data1.shape[1]), dtype= numpy.uint8)
>>> g2 = m.grid('GreenBand')
>>> f2 = g2.field('Green Radiance')
>>> data2 = f2.read(r).data()
>>> sigma2 = numpy.tile((numpy.linspace(0.1,0.6,data2.shape[0])), (data2.shape[1],1)).transpose()
>>> sigma2 = sigma2.astype(numpy.float32)
>>> mask2 = numpy.ones((data2.shape[0],data2.shape[1]), dtype= numpy.uint8)
>>> mapinfo = f2.read(r).mapinfo()
>>> sizefactor = 2
>>> regr_coeff, regr_mapinfo = regr.coeff_calc(data1, mask1, data2, sigma2, mask2, sizefactor, mapinfo)
>>> regr.apply_regression(data1, mask1, mapinfo, regr_coeff, regr_mapinfo)
(array([[ 0.,  0.,  0., ...,  0.,  0.,  0.],
       [ 0.,  0.,  0., ...,  0.,  0.,  0.],
       [ 0.,  0.,  0., ...,  0.,  0.,  0.],
       ...,
       [ 0.,  0.,  0., ...,  0.,  0.,  0.],
       [ 0.,  0.,  0., ...,  0.,  0.,  0.],
       [ 0.,  0.,  0., ...,  0.,  0.,  0.]], dtype=float32), 
array([[0, 0, 0, ..., 0, 0, 0],
       [0, 0, 0, ..., 0, 0, 0],
       [0, 0, 0, ..., 0, 0, 0],
       ...,
       [0, 0, 0, ..., 0, 0, 0],
       [0, 0, 0, ..., 0, 0, 0],
       [0, 0, 0, ..., 0, 0, 0]], dtype=uint8))
\end{verbatim}
\end{methoddesc} 

\begin{methoddesc}{resample_reg_coeff}{}
    Resamples regression coefficients at each pixel.

\begin{verbatim}
>>> r = MtkRegion(37, 50, 60)
>>> m = MtkFile('../Mtk_testdata/in/MISR_AM1_GRP_ELLIPSOID_GM_P037_O029058_AA_F03_0024.hdf')
>>> g1 = m.grid('BlueBand')
>>> f1 = g1.field('Blue Radiance')
>>> data1 = f1.read(r).data()
>>> mask1 = numpy.ones((data1.shape[0],data1.shape[1]), dtype= numpy.uint8)
>>> g2 = m.grid('GreenBand')
>>> f2 = g2.field('Green Radiance')
>>> data2 = f2.read(r).data()
>>> sigma2 = numpy.tile((numpy.linspace(0.1,0.6,data2.shape[0])), (data2.shape[1],1)).transpose()
>>> sigma2 = sigma2.astype(numpy.float32)
>>> mask2 = numpy.ones((data2.shape[0],data2.shape[1]), dtype= numpy.uint8)
>>> mapinfo = f2.read(r).mapinfo()
>>> sizefactor = 2
>>> regr_coeff, regr_mapinfo = regr.coeff_calc(data1, mask1, data2, sigma2, mask2, sizefactor, mapinfo)
>>> target_map_info = MtkRegion(37, 50, 60).snap_to_grid(37, 1100)
>>> regr.resample_reg_coeff(regr_coeff, regr_mapinfo, target_map_info)
<MisrToolkit.MtkRegCoeff object at 0x10a623a30>
\end{verbatim}
\end{methoddesc}                    

\section{Containers \label {containers}}

  All the classes in this sections are returned from other functions.

\subsection{MtkBlockCorners \label{mtkblockcorners}}

\begin{classdesc*}{MtkBlockCorners}
  Block Corners.
\end{classdesc*}

\begin{memberdesc}[int]{block}
  1-based indexed tuple of MtkGeoBlock.
\end{memberdesc}

\begin{memberdesc}[int]{end_block}
  End block number.
\end{memberdesc}

\begin{memberdesc}[int]{path}
  Path number.
\end{memberdesc}

\begin{memberdesc}[int]{start_block}
  Start block number.
\end{memberdesc}

\subsection{MtkGeoBlock \label{mtkgeoblock}}

\begin{classdesc*}{MtkGeoBlock}
  Geographic Block Coordinates.
\end{classdesc*}

\begin{memberdesc}[int]{block}
  Block number.
\end{memberdesc}

\begin{memberdesc}[MtkGeoCoord]{ctr}
  MtkGeoCoord containing center coordinate.
\end{memberdesc}

\begin{memberdesc}[MtkGeoCoord]{llc}
  MtkGeoCoord containing lower left coordinate.
\end{memberdesc}

\begin{memberdesc}[MtkGeoCoord]{lrc}
  MtkGeoCoord containing lower right coordinate.
\end{memberdesc}

\begin{memberdesc}[MtkGeoCoord]{ulc}
  MtkGeoCoord containing upper left coordinate.
\end{memberdesc}

\begin{memberdesc}[MtkGeoCoord]{urc}
  MtkGeoCoord containing upper right coordinate.
\end{memberdesc}


\subsection{MtkGeoCoord \label{mtkgeocoord}}

\begin{classdesc*}{MtkGeoCoord}
  Geographic Coordinates.
\end{classdesc*}

\begin{memberdesc}[float]{lat}
  Latitude in decimal degrees.
\end{memberdesc}

\begin{memberdesc}[float]{lon}
  Longitude in decimal degrees.
\end{memberdesc}


\subsection{MtkGeoRegion \label{mtkgeoregion}}

\begin{classdesc*}{MtkGeoRegion}
  Geometric Region.
\end{classdesc*}

\begin{memberdesc}[MtkGeoCoord]{ctr}
  MtkGeoCoord containing center coordinate.
\end{memberdesc}

\begin{memberdesc}[MtkGeoCoord]{llc}
  MtkGeoCoord containing lower left coordinate.
\end{memberdesc}

\begin{memberdesc}[MtkGeoCoord]{lrc}
  MtkGeoCoord containing lower right coordinate.
\end{memberdesc}

\begin{memberdesc}[MtkGeoCoord]{ulc}
  MtkGeoCoord containing upper left coordinate.
\end{memberdesc}

\begin{memberdesc}[MtkGeoCoord]{urc}
  MtkGeoCoord containing upper right coordinate.
\end{memberdesc}


\subsection{MtkSomCoord \label{mtksomcoord}}

\begin{classdesc*}{MtkSomCoord}
  SOM Coordinates.
\end{classdesc*}

\begin{memberdesc}[float]{x}
  SOM X coordinate.
\end{memberdesc}

\begin{memberdesc}[float]{y}
  SOM Y coordinate.
\end{memberdesc}

\subsection{MtkSomRegion \label{mtksomregion}}

\begin{classdesc*}{MtkSomRegion}
  SOM Region.
\end{classdesc*}

\begin{memberdesc}[MtkSomCoord]{ctr}
  MtkSomCoord containing center coordinate.
\end{memberdesc}

\begin{memberdesc}[MtkSomCoord]{lrc}
  MtkSomCoord containing lower right coordinate.
\end{memberdesc}

\begin{memberdesc}[int]{path}
  Path number.
\end{memberdesc}

\begin{memberdesc}[MtkSomCoord]{ulc}
  MtkSomCoord containing upper left coordinate.
\end{memberdesc}

\subsection{MtkProjParam \label{mtkprojparam}}

\begin{classdesc*}{MtkProjParam}
  This object contains projection parameters.
\end{classdesc*}

\begin{memberdesc}[tuple]{lrc}
  Lower right corner.
\end{memberdesc}

\begin{memberdesc}[int]{nblock}
  Number of blocks.
\end{memberdesc}

\begin{memberdesc}[int]{nline}
  Number of lines.
\end{memberdesc}

\begin{memberdesc}[int]{nsample}
  Number of samples.
\end{memberdesc}

\begin{memberdesc}[int]{path}
  Path number.
\end{memberdesc}

\begin{memberdesc}[long]{projcode}
  Projcode.
\end{memberdesc}

\begin{memberdesc}[tuple]{projparam}
  Projection parameters.
\end{memberdesc}

\begin{memberdesc}[int]{resolution}
  Resolution.
\end{memberdesc}

\begin{memberdesc}[tuple]{reloffset}
  Reloffset.
\end{memberdesc}

\begin{memberdesc}[long]{spherecode}
  Sphere code.
\end{memberdesc}

\begin{memberdesc}[tuple]{ulc}
  Upper left corner.
\end{memberdesc}

\begin{memberdesc}[long]{zonecode}
  Zone code.
\end{memberdesc}

\subsection{MtkRegCoeff \label{mtkregcoeff}}

\begin{classdesc*}{MtkRegCoeff}
  Contains regression coefficient information.

\begin{verbatim}
>>> regr_coeff = MtkRegCoeff()
\end{verbatim}

\begin{verbatim}
>>> regr_coeff, regr_mapinfo = regr.coeff_calc(data1, mask1, data2, sigma2, mask2, sizefactor, mapinfo)
\end{verbatim}
\end{classdesc*}

\begin{methoddesc}{valid_mask}{}
  Returns a NumPy of valid mask for regression coefficients.

\begin{verbatim}
>>> regr_coeff.valid_mask()
array([[0, 0, 0, ..., 0, 0, 0],
       [0, 0, 0, ..., 0, 0, 0],
       [0, 0, 0, ..., 0, 0, 0],
       ...,
       [0, 0, 0, ..., 0, 0, 0],
       [0, 0, 0, ..., 0, 0, 0],
       [0, 0, 0, ..., 0, 0, 0]], dtype=uint8)
\end{verbatim}
\end{methoddesc}

\begin{methoddesc}{slope}{}
  Returns a NumPy of slopes for regression coefficients.

\begin{verbatim}
>>> regr_coeff.slope()
array([[ 0.,  0.,  0., ...,  0.,  0.,  0.],
       [ 0.,  0.,  0., ...,  0.,  0.,  0.],
       [ 0.,  0.,  0., ...,  0.,  0.,  0.],
       ...,
       [ 0.,  0.,  0., ...,  0.,  0.,  0.],
       [ 0.,  0.,  0., ...,  0.,  0.,  0.],
       [ 0.,  0.,  0., ...,  0.,  0.,  0.]], dtype=float32)
\end{verbatim}
\end{methoddesc}

\begin{methoddesc}{intercept}{}
  Returns a NumPy of intercepts for regression coefficients.

\begin{verbatim}
>>> regr_coeff.intercept()
array([[ 0.,  0.,  0., ...,  0.,  0.,  0.],
       [ 0.,  0.,  0., ...,  0.,  0.,  0.],
       [ 0.,  0.,  0., ...,  0.,  0.,  0.],
       ...,
       [ 0.,  0.,  0., ...,  0.,  0.,  0.],
       [ 0.,  0.,  0., ...,  0.,  0.,  0.],
       [ 0.,  0.,  0., ...,  0.,  0.,  0.]], dtype=float32)
\end{verbatim}
\end{methoddesc}

\begin{methoddesc}{correlation}{}
  Returns a NumPy of correlation for regression coefficients.

\begin{verbatim}
>>> regr_coeff.correlation()
array([[ 0.,  0.,  0., ...,  0.,  0.,  0.],
       [ 0.,  0.,  0., ...,  0.,  0.,  0.],
       [ 0.,  0.,  0., ...,  0.,  0.,  0.],
       ...,
       [ 0.,  0.,  0., ...,  0.,  0.,  0.],
       [ 0.,  0.,  0., ...,  0.,  0.,  0.],
       [ 0.,  0.,  0., ...,  0.,  0.,  0.]], dtype=float32)
\end{verbatim}
\end{methoddesc}

\section{Functions \label {funcs}}

\subsection{CoordQuery \label{coordquery}}

\begin{funcdesc}{bls_to_latlon}{path, resolution_meters, block, line, sample}
  Convert from Block, Line, Sample, to Latitude and Longitude in Decimal Degrees.

\begin{verbatim}
>>> bls_to_latlon(189, 275, 47, 12.5, 50.5)
(55.161373, 16.435317)
\end{verbatim}
\end{funcdesc}

\begin{funcdesc}{bls_to_somxy}{path, resolution_meters, block, line, sample}
  Convert from Block, Line, Sample, to SOM Coordinates.

\begin{verbatim}
>>> bls_to_somxy(230, 1100, 69, 100.2, 89.9)
(17145919.996, 222089.993)
\end{verbatim}
\end{funcdesc}

\begin{funcdesc}{latlon_to_bls}{path, resolution_meters, lat, lon}
  Convert decimal degrees latitude and longitude to block, line, sample.

\begin{verbatim}
>>> latlon_to_bls(160, 1100, 57.1, 65.7)
(45, 19.5214, 207.8861)
\end{verbatim}
\end{funcdesc}

\begin{funcdesc}{latlon_to_somxy}{path, lat, lon}
  Convert decimal degrees latitude and longitude to SOM X, SOM Y.

\begin{verbatim}
>>> latlon_to_somxy(160, 57.1, 65.7)
(13677973.731, 686274.716)
\end{verbatim}
\end{funcdesc}

\begin{funcdesc}{path_block_range_to_block_corners}{path, start_block, end_block}
  Compute block corner coordinates in decimal degrees of latitude and longitude for a given path and block range.

\begin{verbatim}
>>> path_block_range_to_block_corners(37, 50, 53)
<MisrToolkit.MtkBlockCorners object at 0x189f800>
\end{verbatim}
\end{funcdesc}

\begin{funcdesc}{path_to_projparam}{path, resolution}
  Get projection parameters.

\begin{verbatim}
>>> path_to_projparam(160, 275)
<MisrToolkit.MtkProjParam object at 0x1897a00>
\end{verbatim}
\end{funcdesc}

\begin{funcdesc}{somxy_to_bls}{path, resolution_meters, somx, somy}
  Convert SOM X, SOM Y to block, line, sample.

\begin{verbatim}
>>> somxy_to_bls(230, 1100, 17145920.0, 222090.0)
(69, 100.2, 89.899)
\end{verbatim}
\end{funcdesc}

\begin{funcdesc}{somxy_to_latlon}{path, somx, somy}
  Convert SOM X, SOM Y to decimal degrees latitude and longitude.

\begin{verbatim}
>>> somxy_to_latlon(230, 17145920.0, 222090.0)
(26.7376, -54.1496)
\end{verbatim}
\end{funcdesc}


\subsection{FileQuery \label{filequery}}

\begin{funcdesc}{find_file_list}{searchdir, product, camera, path, orbit, version}
  Find files in directory tree, using regular expressions.

\begin{verbatim}
>>> find_file_list('../Mtk_testdata/in/', 'GRP_ELLIPSOID_GM', '.*', '037', '029058', 'F03_0024')
['MISR_AM1_GRP_ELLIPSOID_GM_P037_O029058_AA_F03_0024.hdf', '../Mtk_testdata/in/MISR_AM1_GRP_ELLIPSOID_GM_P037_O029058_AN_F03_0024.hdf']
\end{verbatim}
\end{funcdesc}

\begin{funcdesc}{make_filename}{basedir, product, camera, path, orbit, version}
  Given a base directory, product, camera, path, orbit, version make file name.

\begin{verbatim}
>>> make_filename('../Mtk_testdata/in/', 'GRP_ELLIPSOID_GM', 'AA', 37, 29058, 'F03_0024')
'../Mtk_testdata/in/MISR_AM1_GRP_ELLIPSOID_GM_P037_O029058_AA_F03_0024.hdf'
\end{verbatim}
\end{funcdesc}

\subsection{OrbitPath \label{orbitpath}}

\begin{funcdesc}{latlon_to_path_list}{lat, lon}
  Get list of paths that cover a particular latitude and longitude.

\begin{verbatim}
>>> latlon_to_path_list(66.121646, 89.263022)
[7, 8, 9, 10, 11, 12, 13, 14, 146, 147, 148, 149, 150, 151, 152, 153, 154]
\end{verbatim}
\end{funcdesc}

\begin{funcdesc}{orbit_to_path}{orbit}
  Given orbit number return path number.

\begin{verbatim}
>>> orbit_to_path(29058)
37
\end{verbatim}
\end{funcdesc}

\begin{funcdesc}{orbit_to_time_range}{orbit}
  Given orbit number return time range.  Time format: YYYY-MM-DDThh:mm:ssZ (ISO 8601)

\begin{verbatim}
>>> orbit_to_time_range(32467)
('2006-01-24T19:56:53Z', '2006-01-24T21:35:46Z')
\end{verbatim}
\end{funcdesc}

\begin{funcdesc}{path_time_range_to_orbit_list}{path, start, end}
  Given path and time range return list of orbits on path.  Time format: YYYY-MM-DDThh:mm:ssZ (ISO 8601)

\begin{verbatim}
>>> path_time_range_to_orbit_list(37, '2002-02-02T02:00:00Z', '2002-05-02T02:00:00Z')
[11350, 11583, 11816, 12049, 12282, 12515]
\end{verbatim}
\end{funcdesc}

\begin{funcdesc}{time_range_to_orbit_list}{start, end}
  Given start time and end time return list of orbits.  Time format: YYYY-MM-DDThh:mm:ssZ (ISO 8601)

\begin{verbatim}
>>> time_range_to_orbit_list('2005-02-02T02:00:00Z', '2005-02-02T03:00:00Z')
[27271, 27272]
\end{verbatim}
\end{funcdesc}

\begin{funcdesc}{time_to_orbit_path}{time}
  Given time return orbit number and path number.  Time format: YYYY-MM-DDThh:mm:ssZ (ISO 8601)

\begin{verbatim}
>>> time_to_orbit_path('2005-02-02T02:00:00Z')
(27271, 104)
\end{verbatim}
\end{funcdesc}


\subsection{UnitConv \label{unitconv}}

\begin{funcdesc}{dd_to_deg_min_sec}{dd}
  Convert decimal degrees to unpacked degrees, minutes, seconds.

\begin{verbatim}
>>> dd_to_deg_min_sec(65.55)
(65, 32, 60.0)
\end{verbatim}
\end{funcdesc}

\begin{funcdesc}{dd_to_dms}{dd}
    Convert decimal degrees to packed degrees, minutes, seconds.

\begin{verbatim}
>>> dd_to_dms(65.55)
65032060.0
\end{verbatim}
\end{funcdesc}

\begin{funcdesc}{dd_to_rad}{dd}
   Convert decimal degrees to radians.

\begin{verbatim}
>>> dd_to_rad(65.55)
1.144063
\end{verbatim}
\end{funcdesc}

\begin{funcdesc}{deg_min_sec_to_dd}{deg, min, sec}
   Convert unpacked degrees, minutes, seconds to decimal degrees.

\begin{verbatim}
>>> deg_min_sec_to_dd(65, 33, 0.001)
65.55
\end{verbatim}
\end{funcdesc}

\begin{funcdesc}{deg_min_sec_to_dms}{deg, min, sec}
   Convert unpacked Degrees, minutes, seconds to packed.

\begin{verbatim}
>>> deg_min_sec_to_dms(65, 33, 0.001)
65033000.001
\end{verbatim}
\end{funcdesc}

\begin{funcdesc}{deg_min_sec_to_rad}{deg, min, sec}
    Convert unpacked degrees, minutes, seconds to radians.

\begin{verbatim}
>>> deg_min_sec_to_rad(65, 33, 0.001)
1.144063
\end{verbatim}
\end{funcdesc}

\begin{funcdesc}{dms_to_dd}{dms}
    Convert packed degrees, minutes, seconds to decimal degrees.

\begin{verbatim}
>>> dms_to_dd(65033000.010)
65.55
\end{verbatim}
\end{funcdesc}

\begin{funcdesc}{dms_to_deg_min_sec}{dms}
    Convert packed degrees, minutes, seconds to unpacked.

\begin{verbatim}
>>> dms_to_deg_min_sec(65033012.0)
(65, 33, 12.0)
\end{verbatim}
\end{funcdesc}

\begin{funcdesc}{dms_to_rad}{dms}
   Convert packed degrees, minutes, seconds to Radians.

\begin{verbatim}
>>> dms_to_rad(65033000.010)
1.144063
\end{verbatim}
\end{funcdesc}

\begin{funcdesc}{rad_to_dd}{rad}
   Convert radians to decimal degrees.

\begin{verbatim}
>>> rad_to_dd(1.1440634)
66.55
\end{verbatim}
\end{funcdesc}

\begin{funcdesc}{rad_to_deg_min_sec}{rad}
   Convert radians to unpacked degrees, minutes, seconds.

\begin{verbatim}
>>> rad_to_deg_min_sec(1.14406333)
(65, 33, 0.00109)
\end{verbatim}
\end{funcdesc}

\begin{funcdesc}{rad_to_dms}{rad}
    Convert radians to packed degrees, minutes, seconds.

\begin{verbatim}
>>> rad_to_dms(1.14406332)
65032059.999
\end{verbatim}
\end{funcdesc}


\subsection{Util \label{util}}

\begin{funcdesc}{cal_to_julian}{year, month, day, hour, min, sec}
    Convert calendar date to Julian date.

\begin{verbatim}
>>> cal_to_julian(2005, 12, 23, 18, 33, 18)
2453728.2731249998
\end{verbatim}
\end{funcdesc}

\begin{funcdesc}{datetime_to_julian}{datetime}
    Convert date and time in ISO 8601 format to Julian date.

\begin{verbatim}
>>> datetime_to_julian('2005-12-23T18:33:18Z')
2453728.2731249998
\end{verbatim}
\end{funcdesc}

\begin{funcdesc}{julian_to_cal}{julian_date}
    Convert Julian date to calendar date.

\begin{verbatim}
>>> julian_to_cal(2453728.2731249998)
(2005, 12, 23, 18, 33, 18)
\end{verbatim}
\end{funcdesc}

\begin{funcdesc}{julian_to_datetime}{julian_date}
    Convert Julian date to date and time in ISO 8601 format.

\begin{verbatim}
>>> julian_to_datetime(2453728.2731249998)
'2005-12-23T18:33:18Z'
\end{verbatim}
\end{funcdesc}

\begin{funcdesc}{parse_fieldname}{field_name}
    Parses extra dimensions from fieldnames.
\end{funcdesc}

\begin{funcdesc}{version}{}
    MISR Toolkit version.

\begin{verbatim}
>>> version()
'1.2.0'
\end{verbatim}
\end{funcdesc}

\end{document}
